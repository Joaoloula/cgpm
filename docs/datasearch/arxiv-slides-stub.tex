\documentclass{article}

\usepackage{geometry}
  \geometry{
    a4paper,
    total={170mm,257mm},
    left=30mm,
    right=30mm,
    top=20mm,
}

\usepackage[utf8]{inputenc} % allow utf-8 input
\usepackage[T1]{fontenc}    % use 8-bit T1 fonts
\usepackage{hyperref}       % hyperlinks
\usepackage{url}            % simple URL typesetting
\usepackage{booktabs}       % professional-quality tables
\usepackage{amsfonts}       % blackboard math symbols
\usepackage{nicefrac}       % compact symbols for 1/2, etc.
\usepackage{microtype}      % microtypography

\usepackage[labelfont=bf]{caption}
\usepackage[subrefformat=parens]{subcaption}
\usepackage[usenames,dvipsnames,svgnames]{xcolor}
\usepackage{algorithmicx}
\usepackage{algorithm}
\usepackage[noend]{algpseudocode}
\usepackage{amsfonts}
\usepackage{amsmath}
\usepackage{amssymb}
\usepackage{bm}
\usepackage{bbm}
\usepackage{chngcntr}
\usepackage{mdframed}
\usepackage{graphicx}
\usepackage{parskip}
\usepackage{placeins}
\usepackage{soul}
\usepackage{enumerate}

% fonts
\renewcommand{\rmdefault}{ptm}
\renewcommand{\sfdefault}{phv}

\newcommand*\Let[2]{\State {#1} $\gets$ {#2}}
\newcommand*\Sample[2]{\State {#1} $\sim$ {#2}}
\algnewcommand{\LineComment}[1]{\State \(\triangleright\) #1}

\newcommand{\balpha}{\bm\bm{\alpha}}
\newcommand{\bbeta}{\bm\beta}
\newcommand{\bgamma}{\bm\gamma}
\newcommand{\btheta}{\bm\bm{\Theta}}
\newcommand{\bTheta}{\bm\Theta}
\newcommand{\blambda}{\bm\lambda}
\newcommand{\G}{\mathcal{G}}
\newcommand{\R}{\mathbb{R}}
\newcommand{\T}{T}
\newcommand{\s}{\bm{s}}
\newcommand{\x}{\bm{x}}
\newcommand{\X}{\bm{X}}
\newcommand{\y}{\bm{y}}
\newcommand{\Y}{\bm{Y}}
\newcommand{\z}{\bm{z}}
\newcommand{\Z}{\bm{Z}}

\newcommand{\gpm}{\mathcal{G}}
\newcommand{\point}{\mathbf{x}}
\newcommand{\memberindex}{r}
\newcommand{\member}{\mathbf{x}_\memberindex}
\newcommand{\memberj}{x_{[\memberindex, j]}}
\newcommand{\numstates}{N_s}
\newcommand{\numtransitions}{N_t}
\newcommand{\data}{\mathcal{D}}
\newcommand{\datastar}{\data^\star}
\newcommand{\columns}{J}
\newcommand{\givens}{\mathcal{D}_g}
\newcommand{\comparison}{\mathcal{D}_c}
\newcommand{\comparisonindexes}{\mathcal{I}}
\newcommand{\comparisonmember}{\mathbf{x}_i}
\newcommand{\comparisonset}{\set{\x_{[r_i,Q]}}_{i \in \comparisonindexes}}
\newcommand{\score}{\textsc{score}}
\newcommand{\logscore}{\textsc{log score}}
\newcommand{\logpquery}{Q=\set{x_{[r,q_j]}}}
\newcommand{\values}{x_{[i,j]}}
\newcommand{\givenrow}{E=\set{x_[r,e_j]}}

\newcommand{\pop}{\mathcal{P}}
\newcommand{\mV}{\mathcal{V}}
\newcommand{\Vt}{\mathcal{V}_t}
\newcommand{\bx}{\bm{x}}
\newcommand{\bX}{\bm{X}}
\newcommand{\mD}{\mathcal{D}}
\newcommand{\mG}{\mathcal{G}}
\newcommand{\B}{\mathcal{B}}
\newcommand{\bs}{\bm{s}}
\newcommand{\by}{\bm{y}}
\newcommand{\mI}{\mathcal{I}}
\newcommand{\mO}{\mathcal{O}}
\newcommand{\out}{\mathit{out}}
\newcommand{\bZ}{\bm{Z}}
\newcommand{\bz}{\bm{z}}
\newcommand{\bY}{\bm{Y}}
\newcommand{\naturals}{\mathbb{N}}
\newcommand{\bomega}{\bm{\omega}}
\newcommand{\bphi}{\bm{\phi}}
\newcommand{\pg}{p_{\gpm}}
\newcommand\numberthis{\addtocounter{equation}{1}\tag{\theequation}}
\newcommand{\bxr}{\bx_r}
\newcommand{\bxstar}{\bx^\star}
\newcommand{\Dstar}{\text{D}^\star_n}
\newcommand{\schema}{\mathcal{S}}

\newcommand{\variable}[1]{X_{(#1)}}
\newcommand{\measurement}[1]{x_{(#1)}}
\newcommand{\latent}[1]{z_{(#1)}}
\newcommand{\query}[1]{q_{(#1)}}
\newcommand{\bxrn}[1]{\bx_{r_#1}^\star}
\newcommand{\set}[1]{\{{#1}\}}
\newcommand{\bcaption}[2]{\caption{\textbf{#1} {#2}}}

\DeclareMathOperator{\argmin}{arg\,min}
\DeclareMathOperator{\argmax}{arg\,max}
\DeclareMathOperator{\logsumexp}{logsumexp}

\def\indep{\perp\!\!\!\perp}

\newcommand{\red}[1]{\textcolor{red}{#1}}
\newcommand{\blue}[1]{\textcolor{blue}{#1}}

\frenchspacing
\setlength{\floatsep}{5pt plus 1.0pt minus 2.0pt}
\setlength{\textfloatsep}{10pt plus 1.0pt minus 2.0pt}
\setlength{\abovecaptionskip}{2pt plus 0.0pt minus 1.0pt}
% \belowcaptionskip.
\setlength{\abovedisplayskip}{5pt plus 1.0pt minus 2.0pt}
\setlength{\belowdisplayskip}{5pt plus 1.0pt minus 2.0pt}


\begin{document}
\title{Data Search for the DP Mixture}
\author{Leo Casarsa}
\maketitle

\section{Pseudocode for CrossCat Logpdf}

Shortened version of data-type specific logpdf's.

\begin{algorithm}[h]
  \renewcommand{\thealgorithm}{}
  \caption{\texttt{logpdf} for CrossCat}
  \begin{algorithmic}[1]
    \Function{LogPdf}{
      \texttt{GPM}: $\mG$,
      \texttt{query}: $Q := \set{(r_q, \bm{x_q}, z_q)}_{q=q_1}^{q_n}$,
      \texttt{evidence}: $E := \set{(r_e, \bm{x_e}, z_e)}_{e=e_1}^{e_m}$}
      \footnotesize
      \For{$B \in \pi$}
        \Comment{for each block $B$ in the variable partition}
        \Let{$\mathbf{l}$}{$\textsc{Compute-Cluster-Probabilities}(B)$}
          \Comment{retrieve posterior probabilities of proposed clusters}
        \Let{$K$}{$|l|$} 
          \Comment{compute number of proposed clusters}
        \For{$k=1,\ldots,K$}
          \Comment{for each proposed cluster}
          \For{$q \in (Q \cap B)$}
            \Comment{for each query variable in block $B$}
              \If{$\texttt{data-type}(q) = \texttt{binary}$}
                \Comment{for binary variables}
                \Let{$p_G(x_{(r,q)}| \phi_{(z_r=k,q)})$}{$
                  \textsc{BetaBernoulli}(x_{(r,q)};\phi_{(k, q)})$
                } \Comment{compute Beta-Bernoulli density}
              \EndIf
              \If{$\texttt{data-type}(q) = \texttt{nominal}$}
                \Comment{for nominal variables}
                \Let{$p_G(x_{(r,q)}| \phi_{(k,q)})$}{$
                  \textsc{DirichletCategorical}(x_{(r,q)};\phi_{(k, q)})$
                } \Comment{compute Dirichlet-Categorical}
              \EndIf
              \If{$\texttt{data-type}(q) = \texttt{numerical}$}
                \Comment{for numerical variables}
                \Let{$p_G(x_{(r,q)}| \phi_{(k,q)})$}{$
                  \textsc{NIGNormal}(x_{(r,q)};\phi_{(k, q)})$
                } \Comment{compute Normal-Inverse-Gamma-Normal density}
              \EndIf
          \EndFor
        \EndFor
        \Let{$t_B$}{$\displaystyle \sum_{k=1}^K \left(
              \prod_{q \in (Q \cap B)} p_G(x_{(r,q)}| \phi_{(k,q)}) \right) l_k $}
          \Comment{compute density for query variable in $B$}
      \EndFor
      \State \Return{$\displaystyle \sum_{B \in \pi}\log(t_B)$}
        \Comment{return overall log density of query}
    \EndFunction
  \end{algorithmic}
\end{algorithm}

\pagebreak
\section{PseudoCode for CrossCat Incorporate} 
-- Inline math for incorporate

\begin{algorithm}[h]
  \renewcommand{\thealgorithm}{}
  \caption{\texttt{incorporate} for CrossCat}
  \begin{algorithmic}[1]
    \Function{Incorporate}{
      \texttt{GPM}: $\mG$,
      \texttt{rowid}: $r$,
      \texttt{observation}: $(\x_r, \z_r)$}
    \footnotesize
    \Let{$\mG'$}{$\mG$} \Comment{initialize output GPM with $\mG$}
    \Let{$\mD_{\mG'}$}{$\mD_\mG \cup \bm{x_r}$}
      \Comment{assign the row observations into dataset $\mD$}
    \For{$B \in \pi$}
      \Comment{for each block in the variable partition}
      \If{$z_{(r,B)} = \emptyset$} 
      \Comment{if user did not assign a mixture component
        for the observation} 
        \Let{$z_{(r,B)}$}{\textsc{Simulate}$(\mG, r, z_{(r,B)}, \bm{x_r})$}
          \Comment{sample component assignment conditioned on observations}
      \EndIf
      \Let{$\Z_{\mG'}$}{$\Z_\mG \cup z_{(r, B)}$}
        \Comment{include component assignment into row-specific latents}
      \For{$v \in B$}
        \Comment{for each variable in block $B$}
        \Let{$\phi'_{(z_{(r,b)}, v)}$}{$\textsc{UpdateComponentParameters}(z_{(r,B)}, v)$}
          \Comment{compute parameters for $v$ 
            after observing $\bm{x_r}$}
        \Let{$\bm{\Theta}'$}{$\set{\bm{\Theta} \setminus 
            \phi_{(z_{(r,b)}, v)}} \cup \phi'_{(z_{(r,b)}, v)}$}
          \Comment{update the population-level latents}
       \EndFor
    \EndFor
    \State \Return{$\mG'$} \Comment{return GPM with incorporated observations}
    \EndFunction
  \end{algorithmic}
\end{algorithm}

\begin{algorithm}[h]
  \renewcommand{\thealgorithm}{}
  \caption{\texttt{unincorporate} for CrossCat}
  \begin{algorithmic}[1]
    \Function{Unincorporate}{
      \textsc{GPM}: $\mG$,
      \textsc{rowid}: $r$}
    \footnotesize
    \Let{$\mG'$}{$\mG$} \Comment{initialize output GPM with $\mG$}
    \Let{$\mD_{\mG'}$}{$\mD_\mG \setminus \bm{x_r}$}
      \Comment{remove observation $\bm{x_r}$ from the dataset}
      \For{$B \in \pi$}
        \Comment{for each block in the variable partition}
            \Let{$\Z_{\mG'}$}{$\Z_\mG \setminus z_{(r, B)}$}
          \Comment{remove component assignment from row-specific latents}
        \For{$v \in B$}
          \Comment{for each variable in block $B$}
          \Let{$\phi'_{(z_{(r,b)}, v)}$}{$\textsc{UpdateComponentParameters}(z_{(r,B)}, v)$}
            \Comment{compute parameters for $v$ 
              after removing $\bm{x_r}$}
          \Let{$\bm{\Theta}'$}{$\set{\bm{\Theta} \setminus 
              \phi_{(z_{(r,b)}, v)}} \cup \phi'_{(z_{(r,b)}, v)}$}
            \Comment{update the population-level latents}
        \EndFor
      \EndFor
    \State \Return{$\mG'$} \Comment{return GPM with unincorporated row}
    \EndFunction
  \end{algorithmic}
\end{algorithm}

\section{PseudoCode for CrossCat Logpdf-Set}
-- Make context as an argument

\begin{algorithm}[h]
  \renewcommand{\thealgorithm}{}
  \caption{\texttt{logpdf-set} for CrossCat}
  \begin{algorithmic}[1]
    \Function{LogPdf-Set}{
      \texttt{GPM}: $\mG$,
      \texttt{query}: $Q := \set{(r_q, \bm{x_q}, z_q)}_{q=q_1}^{q_n}$,
      \texttt{evidence}: $E := \set{(r_e, \bm{x_e}, z_e)}_{e=e_1}^{e_m}$,
      \texttt{context-variable}: $v$}
    \footnotesize
    \Let{$T$}{$\set{(r, \x_r, \z_r): r \in \mD_\mG \cap (Q \cup E)}$}
      \Comment {store values and category of query and evidence rows already in dataset}
    \For {$r \in T$} \Comment{for query and evidence rows already in the dataset}
      \Let{$\mG'$}{\textsc{Unincorporate}$(\mG, r)$}
        \Comment{unincorporate rows from the GPM before evaluating the logpdf}
    \EndFor

    \Let{$p_j$}{\textsc{Joint-LogPdf-Set}$(\mG', Q \cup E, v, 0)$} 
      \Comment{compute the joint logpdf of query and evidence in the context of $v$}

    \Let{$p_m$}{\textsc{Joint-LogPdf-Set}$(\mG', E, v, 0)$}
      \Comment{compute the marginal logpdf of evidence in the context of $v$}

    \State \Return{$p_j - p_m$}
      \Comment{return $\log p_\mG(Q|E,C) = \log p_\mG(Q,E|C) - \log p_\mG(E|C_e)$}
    \EndFunction

    \vspace{5 pt}
    \noindent\rule{\linewidth}{0.4pt}
    \setcounter{ALG@line}{0}
    {\normalsize \Function{Joint-LogPdf-Set}{
      \texttt{GPM}: $\mG$,
      \texttt{query}: $Q = \set{(r_q, \bm{x_q}, z_q)}_{q=1}^{q_n}$,
      \texttt{context-variable}: $v$,
      \texttt{counter}: $i$}}

    \Let{$Q_v$}{$\set{(r, x, z): (r, x, z) \in Q \cap B_v}$}    
      \Comment{retrieve the query variables restricted to the block to which $v$ belongs}
    \Let{$p_{Q_v}$}{$-\infty$} 
      \Comment{initialize output log probability as $\log p_\mG(Q) = \log 0$}

    \If{$i=|Q|$} \Comment{base case, reached after factoring out all rows in $Q$}
      \Let{$p_{Q_v}$}{$0$} \Comment{base case output, $\log p_\mG(Q)=0$ when $Q=\emptyset$}

    \vspace{5 pt}
    \Else \Comment{recursive case, factorize density using chain rule}
      \If{$z_{q_i}^v \in Q_v$} 
        \Comment{if user specified a component for row $r_{q_i}$}
        \Let{$K$}{$\set{z_{q_i}^v}$}
          \Comment{only possible component is the assigned component}

      \Else \Comment{if user did not specify a component for row $r_{q_i}$}
        \Let{$n_K$}{$|\set{\Z_\mG}|+1$}
          \Comment{get number of possible components}
        \Let{$K$}{$\set{1,\ldots, n_k}$}
          \Comment{store possible component assignments for row $r_{q_i}$}
      \EndIf

      \For {$k \in K$} \Comment{for each possible component}
        \Let{$z_{q_i}^v$}{$k$} \Comment{component $k$ will be assigned to row}
        \Let{$p_i^v$}{$\textsc{LogPdf}
            (\mG, r_{q_i}, (x_{q_i}^v, z_{q_i}^v), \emptyset)$}
            \Comment{evaluate joint density for row $r_{q_i}$}
        \Let{$\mG'$}{\textsc{Incorporate}($\mG, r_{q_i}, (x_{q_i}^v, z_{q_i}^v)$)}
            \Comment{incorporate row and component assignment}
      \Let{$p_i$}{$p_i + \textsc{Joint-LogPdf-Set}(\mG', Q, v, i+1)$}
        \Comment{factor out row $r_{q_i}$ from the joint density of query}
      \Let{$p_{Q_v}$}{$\textsc{Log-Sum-Exp}(p_{Q_v}, p_i)$} 
        \Comment{marginalize out the category assignments}
      \EndFor
    \EndIf
    \State \Return{$p_{Q_v}$}
    \EndFunction
  \end{algorithmic}
\end{algorithm}



\section{PseudoCode for Joint Logpdf-Set}
-- Remove helper

\section{PseudoCode for Relevance Search}
-- Add missing Unincorporate 

\begin{algorithm}[h]
  \renewcommand{\thealgorithm}{}
  \caption{Relevance Search for CrossCat}
  \begin{algorithmic}[1]
  \footnotesize
  \Function{Relevance-Score}{
    \texttt{GPM}: $\mG$,
    \texttt{query}: $Q := (r_q, \bm{x_q})$,
    \texttt{evidence}: $E := \set{(r_e, \bm{x_e})}_{e=e_1}^{e_m}$,
    \texttt{context-variable}: $v$}
  \Let{$p_{H_1}, p_{H_2}$}{$0$} \Comment{initialize the hypotheses log-probabilities}
  \For{$r \in Q \cup E$} \Comment{unincorporate each row before search}
    \Let{$\mG'$}{$\textsc{Unincorporate}(\mG, r)$}
  \EndFor
  \Let{$n_K$}{$|\set{\Z_\mG^v}|+1$} 
    \Comment{compute number of proposed components in the block of  $v$}
  \For{$k_q \in \set{1,\ldots,n_K}$} 
    \Comment{for each proposed component}
    \Let{$\z_q$}{$(r_q,k_q)$} 
      \Comment{name current component assignments for the query row}

    \If{$k_q = n_K$} \Comment{if query has been assigned an empty component}
      \Let{$n_k'$}{$n_k+1$} \Comment{evidence may be assigned another empty component}
    \Else \Comment{otherwise}
      \Let{$n_k'$}{$n_k$} \Comment{no need for another empty component}
    \EndIf

    \For{$k_e \in \set{1,\ldots,n_k'} $} 
      \Comment{for each possible component assignment for the evidence set}
      \Let{$z_e$}{$\set{(r_e,k_e)}$} 
        \Comment{name current component assignment for the evidence set}

        \Let{$p_J$}{$\textsc{LogPdf-Set} (
          \mG, (Q, z_q, E, z_e), v)$}
          \Comment{evaluate density of query and evidence with their respective components}
    
      \If{$k_t = k_q$} 
        \Comment{$H_1$: query and target are in the same component}
        \Let{$p_{H_1}$}{$\textsc{Log-Sum-Exp}(p_{H_1}, p_J)$ }
          \Comment {marginalize over previous component assignments}

      \vspace{5 pt}
      \Else 
        \Comment{$H_2$: query and target are in different categories}
        \Let{$p_{H_2}$}{$\textsc{Log-Sum-Exp}(p_{H_2}, p_J)$ }
          \Comment {marginalize over previous component assignments}
      \EndIf
    \EndFor
  \EndFor 
  \State \Return {$p_{H_1} - p_{H_2}$} \Comment{return the log of the relevance score}
  \EndFunction

  \vspace{5 pt}
  \noindent\rule{\linewidth}{0.4pt}
  \setcounter{ALG@line}{0}
  \Function{Relevance-Search}{
    \texttt{GPM}: $\mG$,
    \texttt{evidence}: $E := \set{(r_e, \bm{x_e})}_{e=e_1}^{e_m}$,
    \texttt{context-variable}: $v$}
  \For{$r \in \mD$} \Comment{for each row in the dataset}
    \Let{$\mathcal{S}[r]$}{$\textsc{Relevance-Score}(\mG, r, E, v)$}
      \Comment{compute and store the score for each row in the dataset}
  \EndFor
  \State \Return{$\textsc{Sorted}(\mathcal{S})$}
    \Comment{return the sorted score for each row} 
  \EndFunction
  \end{algorithmic}
\end{algorithm}

\section{Math for CrossCat generative model}
-- Include the models for binary, categorical and numerical.

\begin{align*}
  \alpha_D &\sim \text{Gamma}(k=1, \theta=1)  & \\
  \vec{\lambda_d} &\sim V_d(\cdot) & \text{for each } d \in [D]  \\
  \pi_D &\sim \text{CRP}(\cdot; \alpha_D, [D]) & \text{partition variables into blocks}\\
  \alpha_R^B &\sim \text{Gamma}(k=1, \theta=1)  & \text{for each block } B \in \pi_D \\
  \pi_R^B &\sim \text{CRP}(\cdot; \alpha_B, [R]) &\text{partition rows for each } B \in \pi_D\\
  x_{(r,d)} &\sim {\displaystyle ML_d(\cdot; \vec{\lambda_d}, \pi_R^{B_d})} 
                                              & \text{for each row } r \in [R] \\
           & & \text{and} \text{for each } d \in [D]
\end{align*}

\section{Math for CrossCat joint density}
--
\begin{align*}
  P(\alpha_D,\set{\vec{\lambda_d}}, \pi_D, \set{\alpha_R^B}, \set{\pi_R^B}, \set{x_{(r,d)}})
    &= P(\alpha_D) P(\pi_D|\alpha_D, [D]) \prod_{d \in [D]} P(\vec{\lambda_d}) 
      P(\set{\alpha_R^{B_d}}) P(\pi_R^{B_d}|\alpha_B) 
      \prod_r P(\set{x_{(r,d)}}| \lambda_d, \pi_R^{B_d})\\
    &= e^{-\alpha_D} \text{CRP}(\pi_D; \alpha_D, [D]) \prod_{d \in [D]} V_d(\vec{\lambda_d}) 
      e^{-\alpha_R^{B_d}} \prod_{r \in [R]} ML_d(\set{x_{(r,d)}}; \lambda_d, \pi_R^{B_d})\\
\end{align*}

\subsection{Marginal Likelihood}

\subsubsection{$d$ is \texttt{binary}}

\begin{align*}
  \beta_i &\sim \text{Gamma}(1,1) & \text{for } i=0,1 \\
  \lambda_d &= (\beta_1, \beta_0) & \\
  \pi_R^{B_d} &= \set{z_1, \ldots, z_r,\ldots, z_R} & \\
  \beta'_{i,k} &= \beta_i + \sum_{\set{r: z_r=k}}\mathbbm{1}\set{x_{(r,d)}=i} 
                                  & \text{for } i=0,1 \\
  ML_d&(x_{(r,d)}; \lambda_d, \pi_R^{B_d}) = \text{Bernoulli}(\frac{\beta'_{1,z_r}}{\beta'_{1,z_r}+\beta_{0,z_r}'}) &  
\end{align*}

I can probably remove the lines for $\lambda, \pi$.

\subsubsection{$d$ is \texttt{nominal}}

\begin{align*}
  \beta_i &\sim \text{Gamma}(1,1) & \text{for } i=0,1,\ldots, l \\
  \lambda_d &= (\beta_0, \ldots, \beta_{n_k}) & \\
  \pi_R^{B_d} &= \set{z_1, \ldots, z_r,\ldots, z_R} & \\
  \beta'_{i,k} &= \beta_i + \sum_{\set{r: z_r=k}}\mathbbm{1}\set{x_{(r,d)}=i} 
                                  & \text{for } i=0,\ldots, l \\
  ML_d&(x_{(r,d)}; \lambda_d, \pi_R^{B_d}) = \text{Categorical}
        (\frac{\beta'_{0,z_r}}{\sum_{i=0}^{l} \beta'_{i,z_r}},\ldots, 
        \frac{\beta'_{l,z_r}}{\sum_{i=0}^{l} \beta'_{i,z_r}}) &  
\end{align*}

\begin{align*}
  ML_d(.|\vec{\lambda_d}) = 
    \begin{cases}
    \text{Bernoulli}(\frac{\beta_1'}{\beta_1'+\beta_0'}) 
      , \text{if } d \text{ is \texttt{binary}}\\
    \text{Categorical}(\frac{\alpha_1'}{\sum_i\alpha_i'},\ldots, 
      \frac{\alpha_K'}{\sum_i\alpha_i'})
      , \text{if } d \text{ is \texttt{nominal}}\\
    \mathcal{T}_{2a_n}(m_n, \frac{b_n(1+V_n}{a_n}), \text{if } d \text{ is \texttt{numerical}}\\
    \end{cases}
\end{align*}

\section{Math for recovering Bayes Sets}
--

\end{document}


%%% Local Variables:
%%% mode: latex
%%% TeX-master: t
%%% End:
